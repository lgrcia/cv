\documentclass[8pt]{article}
\usepackage[footskip=1.5cm, voffset=1cm, textheight=610pt]{geometry}
\usepackage{makecell}
\usepackage{tabularx} 
\usepackage{array}
\usepackage{xcolor}
\usepackage{xifthen}
\usepackage[default]{sourcesanspro}
\usepackage[T1]{fontenc}
\usepackage{setspace}
\usepackage{hyperref}
\usepackage{longtable}
\usepackage{fontawesome}
\usepackage[scaled=0.85]{beramono}

\hypersetup{
    colorlinks=true,
    linkcolor=blue,    
    urlcolor=black,
}

\setlength\extrarowheight{10pt}
\setlength\parindent{0pt}
% \pagenumbering{gobble}

\begin{document}
\small

\begin{center}
    {\huge Lionel J. Garcia} \\
    \vspace{0.25cm}
    born in 1994 in France\\
    \vspace{0.2cm}
    {\faEnvelope\hspace{2pt}  \href{mailto:lionel_garcia@live.fr}{lionel\_garcia@live.fr} \hspace{5pt} \href{https://github.com/lgrcia}{\faGithub\hspace{2pt}lgrcia}}\\
    \vspace{0.1cm}
    {\faGlobe\hspace{2pt} \href{https://lgrcia.github.io}{https://lgrcia.github.io}}
\end{center}

% \newcommand{\block}[5]{
%     \makecell[tr]{\color{black!80}\textbf{#3} \\ {\color{black!50}#1}\\ {\color{black!40}#4}}& 
%     \makecell[tl]{{\color{black!70}\textbf{#2}} \\ \parbox[t]{9cm}{#5}}
% }

\newcommand{\block}[5]{
    \makecell[tr]{\color{black!80}\textbf{#1}\\ {\color{black!50}#3} \\ {\color{black!30}#4}}& 
    \makecell[tl]{{\color{black!70}\textbf{#2}} \\ \parbox[t]{9.2cm}{#5}}
}

% \newcommand{\block}[5]{
%     \makecell[tr]{{\color{black!70}\textbf{#2}}\\{\color{black!50}#1}}& 
%     \makecell[tl]{{\color{black!80}\textbf{#3}} \hspace{0pt} {\color{black!40}#4} \\ \parbox[t]{10cm}{#5}}
% }

\newcommand{\header}[1]{{\color{black!80}\large\MakeUppercase{\textbf{#1}}}}
\newcommand{\subheader}[1]{{\color{black!80}\normalsize\textbf{#1}}}
\newcommand{\blocktitle}[2]{
    \makecell[tr]{{\color{black!70}\textbf{#2}}\\{\color{black!50}#1}}
}

\newcommand{\blockcontent}[3]{
    \makecell[tl]{{\color{black!80}\textbf{#1}} {\color{black!40}#2} \\ \parbox[t]{10cm}{#3}}
}

\newcommand{\blockhead}[1]{\begin{center}\header{#1}\end{center}}
\newcommand{\subblockhead}[1]{\begin{center}\subheader{#1}\end{center}}

\newcommand{\indice}[1]{\color{black!40}\textit{#1}}

\newcommand{\aj}{AJ}
\newcommand{\mnras}{MNRAS}
\newcommand{\aap}{A\&A}
\newcommand{\procspie}{Proc. SPIE}
\newcommand{\baas}{Bulletin of the AAS}
\newcommand{\invited}{\space\space$\star$}

\newenvironment{blocks}[1][1.3]
{\bgroup
\def\arraystretch{#1}
\setstretch{#1}
\small
\begin{tabular}{@{}p{4cm}r}}
{\end{tabular}
\egroup
\vspace{1cm}
} 

\vspace{0.5cm}


\blockhead{education}
\begin{blocks}[1.1]
    \block{2019 - 2023}{PhD in Astronomy}{University of Liège}{Belgium}{Towards the detailed atmospheric characterization of temperate rocky exoplanets \\ \indice{Supervisor: Michaël Gillon (SPECULOOS \& TRAPPIST teams)}}\\
    \block{2016 - 2017}{MSc in Computer Science}{University of Bordeaux}{France}{High performance computing and Image processing}\\
    \block{2014 - 2017}{MSc in Optical Instrumentation}{Institut d'Optique}{France}{Photonics and optical Instrumentation}\\
    \block{2012 - 2014}{BS in Physics}{University of Paris-Sud}{France}{Applied Physics}\\
\end{blocks}

\vspace{-0.3cm}

\blockhead{positions}
\begin{blocks}[1.1]
    \block{2019 - 2021}{Teaching assistant}{University of Liège}{Belgium}{Supervision of tutorial sessions at undergraduate and graduate levels \\ \indice{150h/year -- combined with half-time research}}\\
    \block{2018 (1 year)}{Young Graduate Trainee}{ESA-ESTEC}{Netherlands}{Development of novel strategies to build better spacecraft precursor
    models}\\
    \block{2017 (6 months)}{Technical Student}{CERN}{Switzerland}{Characterization and prototyping of next-generation Beam Wire
    Scanners for the LHC injectors upgrade}\\
    \block{2016 (3 months)}{Trainee}{ESA-ESTEC}{Netherlands}{Development and validation of a CCD cosmic ray impact simulator against
    Gaia in-orbit data}\\
\end{blocks}

\newpage

\blockhead{teaching}
\begin{blocks}[1.2]
    \blocktitle{2019 - 2021}{University of Liège} & \blockcontent{Mechanics 101}{Tutorials - Undergraduate}{From Newtonian mechanics to the study of solids' motion
    \\\indice{Professor: Pierre Dauby}} \\
    {} & \blockcontent{Mathematical modeling for the environment}{Tutorials - Graduate}{Dynamical modeling of populations and their environments
    \\\indice{Professor: Marilaure Grégoire}} \\
    {} & \blockcontent{Astronomical observations}{Tutorials and Lectures - Graduate}{Telescope observations and applications to Astrophysics (including practical sessions at the Oukaimeden observatory, Morocco)
    \\\indice{Professor: Emmanuel Jehin}}\\
\end{blocks}

\blockhead{software developments}
% http://mirrors.ibiblio.org/CTAN/fonts/fontawesome/doc/fontawesome.pdf
\begin{blocks}[1.3]
    \block{}
    {\textit{prose}\hspace{0.1cm}
    {\large\href{https://github.com/lgrcia/prose}{\color{black!30}\faGithub\hspace{2pt}}}
    {\large\href{https://ui.adsabs.harvard.edu/abs/2022MNRAS.509.4817G/abstract}{\color{black!30}\faFileTextO\hspace{2pt}}}}{Image Processing}{Python, LaTeX}{A Python package to build image processing pipelines. Developed for Astronomy to enable transparent research and reproducible products.}
    \\
    \block{}
    {\textit{nuance}\hspace{0.1cm}
    {\large\href{https://github.com/lgrcia/nuance}{\color{black!30}\faGithub\hspace{2pt}}}}
    {Signal Processing}{Python, JAX}{A Python package to detect exoplanetary transits in the presence of stellar variability and instrumental noises.}
    \\
    \block{}
    {\textit{SPECULOOS - portal}\hspace{0.1cm}
    {\large\href{https://lgrcia.github.io/projects/portal/}{\color{black!30}\faGlobe\hspace{2pt}}}}
    {Web application}{HTML-CSS-JS (VueJS)}{A web-based portal to monitor the SPECULOOS transit survey nightly observations (interactive schedule, data visualization, comments and flagging system, diagnostics, and more)}
    \\
    \block{}
    {\textit{SPECULOOS - workflows}\hspace{0.1cm}}
    {Data analysis}{Python, snakemake}{Developement of data analysis workflows for the automatic analysis and reporting of SPECULOOS observations (Manager of the related working group).}
    \\
    \block{}
    {\textit{TRAPPIST - ESO public release}\hspace{0.1cm}}
    {Data analysis}{Python, snakemake, prose}{Reduction and first release of the TRAPPIST telescope photometric products (beginning 2023 with ESO).}
\end{blocks}

\begin{center}
    \color{black!70}
    \subheader{Related skills:}\hspace{0.2cm} Python\space\space C++ \space\space C\space\space Julia\space\space JS - HTML - CSS\space\space LaTeX\space\space git

\end{center}
\vspace{0.4cm}

\vspace{0.4cm}

\newcommand{\publi}[4]{{\color{black!60}#1} & \makecell[lt]{\textbf{#2}\\\rule{0pt}{4pt}{\color{black!60} #3 \hspace{1pt} \textit{#4}}} \\
}

\vspace{0.4cm}
\blockhead{Communications}
\vspace{-0.5cm}
\begin{center}
    \small\color{black!35}
    $\star$ denotes invited
\end{center}

{\footnotesize
\def\arraystretch{1.1}
\setstretch{1.2}
\begin{tabular}{ll}
% \publi{\makecell[lt]{Dec. 2022\invited{}\\Conference}}
%     {The discovery of potentially habitable planets}
%     {\underline{Garcia L. J.}}
%     {Science and Society, (University of Loraine, France)}
\publi{\makecell[lt]{July 2022\\Talk}}
    {HST/WFC3 transmission spectroscopy of the cold rocky planet TRAPPIST-1h}
    {\underline{Garcia L. J.}, Moran S., Rackham B. V. et al.}
    {NAM 2022 (Warwick, UK)}
\publi{\makecell[lt]{July 2022\\Poster}}
    {The bright future of PSF photometry using convolutional neural networks}
    {\underline{Garcia L. J.}}
    {NAM 2022 (Warwick, UK)}
\publi{\makecell[lt]{May 2022\\Poster}}
    {Transmission spectroscopy of the cold rocky planet TRAPPIST-1h}
    {\underline{Garcia L. J.}, Moran S., Rackham B. V. et al.}
    {Exoplanet IV (Las Vegas, USA)}
\publi{\makecell[lt]{May 2020\\Talk}}
    {TRAPPIST-1h transmission spectrum: Knowning the star}
    {\underline{Garcia L. J.}, Moran S., Rackham B. V. et al.}
    {SAG21 symposium (online)}
\publi{\makecell[lt]{Jun. 2019\invited{}\\Poster}}
    {specphot: a suite for SPECULOOS data analysis}
    {\underline{Garcia L. J.} \& the SPECULOOS team}
    {TRAPPIST-1 conference (Liège, Belgium)}
\end{tabular}
}


\vspace{1cm}

\blockhead{Grants}
\vspace{-0.1cm}

\begin{blocks}[1.1]
    \block{2021}{FRIA Doctoral scholarship}{F.R.S.–FNRS}{}{2 years full-time research funding}\\
\end{blocks}

\blockhead{publications}
\vspace{-0.5cm}
\begin{center}
    \small\color{black!35}
    First-authored
\end{center}

{\footnotesize
\def\arraystretch{1.1}
\setstretch{1.2}
\begin{longtable}{rl}
    % \publi{2022}
%     {Spectroscopic anatomy of a polar spot on an M4-type star}
%     {\underline{Garcia L. J.} et al.}
%     {in prep.}

\publi{2023}
{\texttt{nuance}: Transit detection in the presence of stellar variability and correlated noise}
{\underline{Garcia L. J.}, Foreman-Mackey, D.}
{\href{https://github.com/lgrcia/paper-nuance/blob/main/latex/ms.pdf}{in prep. for \aap}}
    \publi{2022}{HST/WFC3 transmission spectroscopy of the cold rocky planet TRAPPIST-1h}
    {\underline{Garcia L. J.}, Moran, S.~E., Rackham, B.~V., et al.}
    {\href{https://ui.adsabs.harvard.edu/abs/2022A\&A...665A..19G}{\aap, 665, A19}}

\publi{2022}
    {\texttt{prose}: a Python framework for modular astronomical images processing}
    {\underline{Garcia L. J.}, Timmermans, Mathilde, Pozuelos, Francisco J. et al.}
    {\href{https://ui.adsabs.harvard.edu/abs/2022MNRAS.509.4817G/exportcitation}{MNRAS 509 4817-4828}}

\publi{2018}
    {Validation of a CCD cosmic ray event simulator against Gaia in-orbit data}
    {\underline{Garcia L.}, Prod'homme T., Lucsanyi D. et al.}
    {\href{http://doi.org/10.1117/12.2314090}{Proc. SPIE 10709}}

\end{longtable}
}

\begin{center}
    \color{black!50}
    Other collaborations
\end{center}

{\footnotesize
\def\arraystretch{1.1}
\setstretch{1}
\begin{longtable}{ll}
\publi{2022}{Two temperate super-Earths transiting a nearby late-type M dwarf}
{Delrez, L., Murray, C.~A., Pozuelos, F.~J., et al. (including Garcia L.~J.)}
{\href{https://ui.adsabs.harvard.edu/abs/2022A\&A...667A..59D}{\aap, 667, A59}}

\publi{2022}{SPECULOOS Northern Observatory: searching for red worlds in the northern skies}
{Burdanov, A.~Y., de Wit, J., Gillon, M., et al. (including Garcia L.~J.)}
{\href{https://ui.adsabs.harvard.edu/abs/2022arXiv220909112B}{arXiv e-prints, arXiv:2209.09112}}

\publi{2022}{Two temperate super-Earths transiting a nearby late-type M dwarf}
{Delrez, L., Murray, C.~A., Pozuelos, F.~J., et al. (including Garcia L.~J.)}
{\href{https://ui.adsabs.harvard.edu/abs/2022arXiv220902831D}{arXiv e-prints, arXiv:2209.02831}}

\publi{2022}{Observation Scheduling and Automatic Data Reduction for the Antarctic telescope, ASTEP+}
{Dransfield, G., Mekarnia, D., Triaud, A.~H.~M.~J., et al. (including Garcia L.~J.)}
{\href{https://ui.adsabs.harvard.edu/abs/2022arXiv220804501D}{arXiv e-prints, arXiv:2208.04501}}

\publi{2022}{TESS discovery of a sub-Neptune orbiting a mid-M dwarf TOI-2136}
{Gan, T., Soubkiou, A., Wang, S.~X., et al. (including Garcia L.~J.)}
{\href{https://ui.adsabs.harvard.edu/abs/2022MNRAS.514.4120G}{\mnras, 514, 4120}}

\publi{2022}{A study of flares in the ultra-cool regime from SPECULOOS-South}
{Murray, C.~A., Queloz, D., Gillon, M., et al. (including Garcia L.~J.)}
{\href{https://ui.adsabs.harvard.edu/abs/2022MNRAS.513.2615M}{\mnras, 513, 2615}}

\publi{2022}{Complex Modulation of Rapidly Rotating Young M Dwarfs: Adding Pieces to the Puzzle}
{G{\"u}nther, M.~N., Berardo, D.~A., Ducrot, E., et al. (including Garcia L.~J.)}
{\href{https://ui.adsabs.harvard.edu/abs/2022AJ....163..144G}{\aj, 163, 144}}

\publi{2022}{TOI-1442 b and TOI-2445 b: two ultra-short period super-Earths around M dwarfs}
{Morello, G., Parviainen, H., Murgas, F., et al. (including Garcia L.~J.)}
{\href{https://ui.adsabs.harvard.edu/abs/2022arXiv220113274M}{arXiv e-prints, arXiv:2201.13274}}

\publi{2022}{TOI-2257 b: A highly eccentric long-period sub-Neptune transiting a nearby M dwarf}
{Schanche, N., Pozuelos, F.~J., G{\"u}nther, M.~N., et al. (including Garcia L.~J.)}
{\href{https://ui.adsabs.harvard.edu/abs/2022A\&A...657A..45S}{\aap, 657, A45}}

\publi{2021}{A large sub-Neptune transiting the thick-disk M4 V TOI-2406}
{Wells, R.~D., Rackham, B.~V., Schanche, N., et al. (including Garcia L.~J.)}
{\href{https://ui.adsabs.harvard.edu/abs/2021A\&A...653A..97W}{\aap, 653, A97}}

\publi{2021}{Six transiting planets and a chain of Laplace resonances in TOI-178}
{Leleu, A., Alibert, Y., Hara, N.~C., et al. (including Garcia L.~J.)}
{\href{https://ui.adsabs.harvard.edu/abs/2021A\&A...649A..26L}{\aap, 649, A26}}

\publi{2021}{SPECULOOS: Ultracool dwarf transit survey. Target list and strategy}
{Sebastian, D., Gillon, M., Ducrot, E., et al. (including Garcia L.~J.)}
{\href{https://ui.adsabs.harvard.edu/abs/2021A\&A...645A.100S}{\aap, 645, A100}}

\publi{2020}{Development of the SPECULOOS exoplanet search project}
{Sebastian, D., Pedersen, P.~P., Murray, C.~A., et al. (including Garcia L.~J.)}
{\href{https://ui.adsabs.harvard.edu/abs/2020SPIE11445E..21S}{\procspie, 11445, 1144521}}

\publi{2020}{{\ensuremath{\pi}} Earth: A 3.14 day Earth-sized Planet from K2's Kitchen Served Warm by the SPECULOOS Team}
{Niraula, P., Wit, J. de ., Rackham, B.~V., et al. (including Garcia L.~J.)}
{\href{https://ui.adsabs.harvard.edu/abs/2020AJ....160..172N}{\aj, 160, 172}}

\publi{2020}{A super-Earth and a sub-Neptune orbiting the bright, quiet M3 dwarf TOI-1266}
{Demory, B.-O., Pozuelos, F.~J., G{\'o}mez Maqueo Chew, Y., et al. (including Garcia L.~J.)}
{\href{https://ui.adsabs.harvard.edu/abs/2020A\&A...642A..49D}{\aap, 642, A49}}

\publi{2020}{GJ 273: on the formation, dynamical evolution, and habitability of a planetary system hosted by an M dwarf at 3.75 parsec}
{Pozuelos, F.~J., Su{\'a}rez, J.~C., de El{\'\i}a, G.~C., et al. (including Garcia L.~J.)}
{\href{https://ui.adsabs.harvard.edu/abs/2020A\&A...641A..23P}{\aap, 641, A23}}

\publi{2020}{Photometry and performance of SPECULOOS-South}
{Murray, C.~A., Delrez, L., Pedersen, P.~P., et al. (including Garcia L.~J.)}
{\href{https://ui.adsabs.harvard.edu/abs/2020MNRAS.495.2446M}{\mnras, 495, 2446}}
\end{longtable}
}

\blockhead{Miscellaneous}
\vspace{-0.2cm}
\begin{center}
{\small Ultra running\space\space - \space\space Illustration\space\space }
\end{center}

\end{document}